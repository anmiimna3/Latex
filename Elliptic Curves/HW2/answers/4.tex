\begin{enumerate}[label=]
    \item $(\Leftarrow)$
        Suppose $X$ is irreducible. And for the sake of contradiction suppose there exists $PQ \in I(X)$, where $P \notin I(X)$ and $Q \notin I(X)$. This means that there exists some $x_1 \in X$ such that $P(x_1) \ne 0$, and also $x_2 \in X$ such that $Q(x_2) \ne 0$. But also since $PQ \in I(X)$, then for any $x \in X$, we have that $PQ(x) = 0$, this means that for any $x \in X$ at least one of the $P(x) = 0$ or $Q(x) = 0$ happens. Let all roots of $P$ which are in $X$ be $X_P$, similarly we define $X_Q$. It is obvious that $X_p \cup X_Q = X$. And $X_P \subsetneq X$ and $X_Q \subsetneq X$. It only remains to show that $X_P$ and $X_Q$ are algebraic sets. Now consider the ideal $\braket{I(X), P}$. Sicne $X_P \subset X$, then $V(\braket{I(X), P}) = X_P$. Similarly $X_Q$ is also algebraic. This shows that $X$ is reducible, which is a contradiction. Then either $P \in I(X)$ or $Q \in I(X)$, which shows that $I(X)$ is prime.
    \item $(\Rightarrow)$
        Suppose $I(X)$ is prime. And for the sake of contradiction suppose that $X = X_1 \cup X_2$ where $X_1 , X_2 \ne X$, and both $X_1$ and $X_2$ are algebraic sets. Consider $I(X_1)$. It has some element $P$ where for some $y \in X - X_1$, $P(y) \ne 0$. Similarly we can find $Q \in I(X_2)$ where for some $z \in X - X_2$, $Q(z) \ne 0$. Now consider $PQ$. Since $X_1 \cup X_2 = X$, then $PQ \in I(X)$. But because of $y$ and $z$ it is obvious that $P, Q \notin I(X)$. Which shows that $I(X)$ is not prime, which is a contradiction, thus $X$ is irreducible.
\end{enumerate}