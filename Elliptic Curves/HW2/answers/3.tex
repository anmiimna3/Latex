\begin{enumerate}[label=]
    \item 
        Suppose we have $f = hg$ and $f$ is of degree $d$, where $h = h_0 + h_1 + \dots + h_n$ and $g = g_0 + g_1 + \dots + g_m$ and $m + n = d$, And $g_i$ and $h_i$ are homogeneous form of degree $i$, or 0.
        We know that product of homogeneous forms is also homogeneous. Suppose $x$ and $y$ are the smallest number that $h_x$ and $g_y$ are nonzero. Thus the smallest homogeneous for in $gh$ is of degree $x + y$. But since $f$ is homogeneous, then all terms in $f$ are of degree $x + y$. Since $g_m$ and $h_n$ are both nonzero (since we supposed $h$ is of degree $n$ and $g$ is of degree $m$), then we have a form with degree $m + n$, and thus $m + n = x + y$. This shows that $m$ and $n$ are the smallest numbers that $h_n$ and $g_m$ are nonzero, therefore $g = g_m$ and $h = h_n$, and both $f$ and $g$ are homogeneous.
\end{enumerate} 