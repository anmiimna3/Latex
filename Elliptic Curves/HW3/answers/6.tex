\begin{enumerate}[label=]
    \item 
        We can write $Q_8$ with order $\{1, -1, i, -i, j, -j, k, -k\}$. Since $i$ and $j$ are generators of $Q_8$ then we just have to check these two elements:
        \begin{gather*}
            i(1) = i.1 = i = 3 \\
            i(2) = i.(-1) = -i = 4 \\
            i(3) = i.(i) = -1 = 2 \\
            i(4) = i.(-i) = 1 = 1 \\
            i(5) = i.(j) = k = 7 \\
            i(6) = i.(-j) = -k = 8 \\
            i(7) = i.(k) = -j = 6 \\
            i(8) = i.(-k) = j = 5
        \end{gather*}
        This shows that $i$ operates like $\delta = (1 \ 3 \ 2 \ 4)(5 \ 7 \ 6 \ 8)$ in $S_8$. Similarly:
        \begin{gather*}
            j(1) = j.(1) = j = 5 \\
            j(2) = j.(-1) = -j = 6 \\
            j(3) = j.(i) = -k = 8 \\
            j(4) = j.(-i) = k = 7 \\
            j(5) = j.(j) = -1 = 2 \\
            j(6) = j.(-j) = 1 = 1 \\
            j(7) = j.(k) = i = 3 \\
            j(8) = j.(-k) = -i = 4
        \end{gather*}
        This shows that $j$ operates like $\sigma = (1 \ 5 \ 2 \ 6)(3 \ 8 \ 4 \ 7)$ in $S_8$. \\
        Therefore $Q_8 = \braket{i, j}$ is isomorphic to $\braket{\delta, \sigma}$ in $S_8$.
\end{enumerate}