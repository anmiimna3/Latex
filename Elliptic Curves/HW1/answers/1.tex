\begin{enumerate}[label=\ilabel]
    \item 
        \begin{enumerate}
            \item 
                Suppose $a = (a_1, a_2, \dots, a_n) \in V(J_1) \cup V(J_2)$. WLOG let $a \in V(J_1)$. This means that for any $P \in K[x_1, \dots, x_n]$ we have $P(a) = 0$.
                Now for any $Q \in J_1 \cap J_2$, we have $Q \in J_1$, thus $Q(a) = 0$ which implies $a \in V(J_1 \cap J_2)$. This shows that :
                \begin{gather*}
                    V(J_1) \cup V(J_2) \subseteq V(J_1 \cap J_2)
                \end{gather*}
                Now let $a \in V(J_1 \cap J_2)$. And for the sake of contradiction suppose there exists $P \in J_1$ and $Q \in J_2$ such that $P(a) \ne 0$ and $Q(a) \ne 0$. Note that $PQ \in J_1$ and $PQ \in J_2$. Therefore $PQ \in J_1 \cap J_2$. Since $a \in V(J_1 \cap J-2)$, then we have $PQ(a) = P(a) Q(a) = 0$. This shows that either $P(a) = 0$ or $Q(a) = 0$, which is a contradiction. Tehrefore at least for one of the $J_1$ and $J_2$, $a$ is a root for all polynomials in that ideal. Let that be $J_1$, thus we have $a \in V(J_1) \subseteq V(J_1) \cup V(J_2)$. Now we have $V(J_1 \cap J_2) \subseteq V(J_1) \cup V(J_2)$, hence:
                \begin{gather*}
                    V(J_1) \cup V(J_2) = V(J_1 \cap J_2)
                \end{gather*}
                Again suppose $a \in V(J_1) \cup V(J_2)$, WLOG $a \in V(J_1)$. Consider a member of $J_1J_2$: 
                \begin{gather*}
                    P_i \in J_1, Q_i \in J_2 \\
                    R = P_1 Q_1 + P_2 Q_2 + \dots P_l Q_l \in J_1J_2 \\
                    \begin{split}
                        \implies R(a) &= P_1(a) Q_1(a) + \dots + P_l(a) Q_l(a) \\
                        & = 0 Q_1(a) + \dots + 0 Q_l(a) = 0
                    \end{split}
                \end{gather*}
                And since $R$ was an arbitrary member of $J_1J_2$, then $a \in V(J_1J_2)$. Then $V(J_1) \cup V(J_2) \subseteq V(J_1J_2)$. Conversely suppose $a \in V(J_1J_2)$. 
                And again for the sake of contradiction suppose $P \in J_1$ and $Q \in J_2$ such that $P(a) \ne 0$ and $Q(a) \ne 0$. Since $PQ \in J_1J_2$, then $P(a) Q(a) = 0$. Which is a contradiction. Therefore at least one of $J_1$ and $J_2$ have $a$ as a root. Thus $a \in V(J_1)$ or $a \in V(J_2)$, which means $a \in V(J_1) \cup V(J_2)$. This shows that $V(J_1J_2) \subseteq V(J_1) \cup V(J_2)$. This proves that:
                \begin{gather*}
                    V(J_1) \cup V(J_2) = V(J_1J_2)
                \end{gather*}
            \item
                Let $a \in V(\sum_{\lambda \in I} J_\lambda)$. Now for any $P \in J_i$, since $P + 0 + 0 + \dots + 0 \in \sum_{\lambda \in I} J_\lambda$ then $P(a) = 0$. Therefore for any $P \in J_i$ we have $P(a) = 0$, Therefore $a \in V(J_i)$, hence $a \in \bigcap_{\lambda \in I} V(J_\lambda)$, therefore $V(\sum_{\lambda \in I} J_\lambda ) \subseteq \bigcap_{\lambda \in I} V(J_\lambda)$.
                Now suppose $a \in \bigcap_{\lambda \in I} V(J_\lambda)$. Now any element in $\sum_{\lambda \in I} J_\lambda$ is of the form $\sum_{\lambda \in L} P_l$ where $P_\lambda \in J_l$ and $L$ is a finite subset of $I$. Now since $a \in V(J_\lambda)$ for any $\lambda$, then $P_\lambda(a) = 0$ for any $\lambda \in L$. Then we can see that $\sum_{\lambda \in L} P_\lambda(a) = 0$. Thus $a \in V(\sum_{\lambda \in I} J_\lambda)$:
                \begin{gather*}
                    V(\sum_{\lambda \in I} J_\lambda) = \bigcap_{\lambda \in I} V(J_\lambda)
                \end{gather*}
        \end{enumerate}
    \item 
        We know that union of two algebraic sets is algebraic, and intersection of any family of algebraic sets is also algebraic. Now since these are the closed sets, then complements of these sets are open sets in this topology.
        To show that indeed is a topology we need to show that $A^n(K)$ and $\emptyset$ are in $T$. This is trivial since complements of these sets are $\emptyset$ and $A^n(K)$ which both are algebraic sets. Also union of any collections of open sets is also open since for complements we must have intersection of any family of closed sets is also closed, which we talked about earlier. Also intersection of any two open sets must be open, which means union of any two closed sets must be closed, which is true for algebraic sets. Hence this is a topology.
\end{enumerate}