\begin{enumerate}[label=\ilabel]
    \item 
        Let $P \in \sqrt{J}$. This means that $P^r \in J$ for some $r \in \NN$. 
        \begin{gather*}
            \forall a \in V(J): P^r(a) = 0 \implies P(a) = 0 \\
            \implies P \in I(V(J))
        \end{gather*}
        This shows that $\sqrt{J} \subseteq I(V(J))$. 
        % Now suppose $K$ is algebraicly closed. Let $P \in I(V(J))$. This means that for any $a \in V(J)$, $P(a) = 0$. 
        
    \item
        Let $P \in \sqrt{IJ}$. Thus $P^r \in IJ$ for some $r \in \NN$. This means that $P^r = P_1 Q_1 + \dots + P_k Q_k$ for some $P_i \in I$ and $Q_i \in J$. Now for any $i$, $P_i Q_i \in I$ and also $P_i Q_i \in J$. Therefore $P^r \in I$ and $P^r \in J$. Thus $P^r \in I \cap J$, which means that $P \in \sqrt{I \cap J}$. Hence $\sqrt{IJ} \subseteq \sqrt{I \cap J}$. Now suppose $P \in \sqrt{I \cap J}$. Then $P^r \in I \cap J$ for some $r \in \NN$. Which means $P^r \in I$ and $P^r \in J$. Then $P^r \times P^r \in IJ$. Therefore $P \in \sqrt{IJ}$. Hence we have $\sqrt{IJ} = \sqrt{I \cap J}$.
        % Now suppose $\sqrt{J} \subseteq \sqrt{I}$. We show that this implies 
\end{enumerate}