\begin{enumerate}[label=\ilabel]
    \item
        \begin{gather*}
            1(B) = B \implies 1 \in G_B \\
            g \in G_B: \inv g(B) = \inv g(g(B)) = 1(B) = B \implies \inv g \in G_B \\
            g_1, g_2 \in G_B: g_1g_2(B) = g_1(g_2(B)) = g_1(B) = B  \implies g_1g_2 \in G_B
        \end{gather*}
        Therefore $G_B$ is a sub-group of $G$.
    \item  
        For $S_4$ on $\{1, 2, 3, 4\}$ we just have to check subsets with 2 and 3 members.
        If $B$ has two elements, $B = \{1, 2\}$ then for $\sigma = (1 \ 3) \in S_4$, $\sigma(B) = \{2, 3\}$. Which proves that $B$ is not a block.
        If $B$ has three elements, $B = \{1, 2, 3\}$. then for $\sigma = (3 \ 4) \in S_4$, $\sigma(B) = \{1, 2, 4\}$. Which again proves that $B$ is not a block.
        Therefore this action is primitive. \\
        For $D_8$ on square. We can see that $A = \{A_1, A_3\}$ is a block.
        \begin{center}
            \tikz{
                \node at (-1, 2) (1) {$A_1$};
                \node at (1, 2) (2) {$A_2$};
                \node at (1, 0) (3) {$A_3$};
                \node at (-1, 0) (4) {$A_4$};

                \path 
                    (1) edge (2)
                    (2) edge (3) 
                    (3) edge (4)
                    (4) edge (1);
            }
        \end{center} 
        Since generators of $D_8$ are $r$ and $s$ we just have to check that these two act on $A$ correctly. Since $\overline{A} = \{A_2, A_4\}$ is an identical block to $A$.
        \begin{gather*}
            r(A) = \{A_2, A_4\} \implies \overline{A} \cap A = \varnothing \\
            s(A) = A
        \end{gather*}
        We can also say the same about $\overline{A}$ as well. \newline
        This proves that $A$ is a non-trivial block. Therefore this action is not primitive.
\end{enumerate}