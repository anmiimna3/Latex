\begin{enumerate}[label=\ilabel]
    \item 
        We need to introduce a homomorphism from $V$ to $U \oplus W$. 
        \begin{center}
            \tikz{
                \node at (0, 2) (1) {$0$};
                \node at (0, 0) (2) {$0$};
                \node at (2, 2) (u1) {$U$};
                \node at (2, 0) (u2) {$U$};
                \node at (4, 2) (v) {$V$};
                \node at (4, 0) (plus) {$U \oplus W$};
                \node at (6, 2) (w1) {$W$};
                \node at (6, 0) (w2) {$W$};
                \node at (8, 2) (3) {$0$};
                \node at (8, 0) (4) {$0$};

                \path [->]
                    (2) edge (u2)
                    (u1) edge [above] node {$\varphi$} (v)
                    (v) edge [above] node {$\sigma$} (w1)
                    (w1) edge (3)
                    (u2) edge [below] node {$p_1$} (plus)
                    (plus) edge [below] node {$p_2$} (w2)
                    (w2) edge (4)
                    (u1) edge [left] node {$id_U$} (u2)
                    (w1) edge [right] node {$id_W$} (w2)
                    (v) edge (plus)
                    (1) edge (u1);
            }
        \end{center}
        by exactness in the sequence we know that:
        \begin{gather*}
            ker(\varphi) = 0 \hspace{1cm}
            Im(\varphi) = ker(\sigma) \hspace{1cm}
            Im(\sigma) = W
        \end{gather*}
        Each of these homomorphisms are uniquely identified with how they behave on a the basis. Define:
        \begin{gather*}
            \begin{split}
                \psi: V &\to U \oplus W  \\
                v &\mapsto \braket{\varphi^{-1}(v), \sigma(v)}
            \end{split}
        \end{gather*}
        Since $ker(\varphi) = 0$, then $\inv \varphi$ is well defined. To show that This means that for any element in $v \in V$, there exist $f_1, f_2, \dots, f_n \in \mathbb{F}$ such that:
        \begin{gather*}
            v = f_1 v_1 + f_2 v_2 + \dots + f_n v_n
        \end{gather*}

\end{enumerate}