\begin{enumerate}[label=\ilabel]
    \item 
        Let $R = \ZZ_6$ and $M = \ZZ_6$. Then number $3$ is not linearly independent since there exists a nonzero solution to the equation:
        \begin{gather*}
            r 3 = 0
        \end{gather*}
        Which is $r = 2$. This shows that the statement is not true for modules.
    \item
        Again we can use the example from the previous part for $\ZZ_6$ over $\ZZ_6$ as module. The set $\{2, 3\}$ is not linearly independent since:
        \begin{gather*}
            0 \times 2 + 4 \times 3 = 0
        \end{gather*}
        But there is no element $x \in \ZZ_6$ such that $2x = 3$.
    \item
        Take $\ZZ$ as a $\ZZ$ module. Notice that $\{2\}$ is a linearly independent set, since there is no nonzero element $x \in \ZZ$ such that $2x = 0$. To show that this set is also maximal, suppose that the set $\{2, a\}$ is linearly independent for some $a \in \ZZ$. Then it can be seen that:
        \begin{gather*}
            a \times 2 + (-2) \times a = 0
        \end{gather*}
        reaching to a contradiction. Now that $\{2\}$ is a maximal independent set, it can be seen that it doesn't generate the whole $\ZZ$ since it only generates even numbers.
    
    \item
        A basis must be linearly independent, but in the previous part, we showed an example of a maximal linearly independent set that did not generate $\ZZ$. Since it was maximal, then it can't be extended to another linearly independent set that generates the space.
    
    \item
        Consider $A = R[x_1, x_2, \dots]$ as a $A$-module where $R$ has 1. Then it is finitely generated by $1_R$. But consider the submodule $\braket{x_1, x_2, \dots}$. This is not finitely generated.
\end{enumerate}