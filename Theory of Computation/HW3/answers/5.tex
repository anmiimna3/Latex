\begin{enumerate}[label=\ilabel]
    \item
        $\beta(J(m, n, q)) = 4(\pi(\pi(m-1, n-1), q-1)) + 3$:
        \begin{gather*}
            \pi(3-1, 4 -1)= 2^2(2\times3+1) - 1 = 27 \\
            \pi(27, 2 - 1) = 2^{27}(2 \times 1 + 1) - 1 = 3 \cdot 2^{27} - 1 \\
            \implies \beta(J(3, 4, 2)) = 4(3 \cdot 2^{27} - 1) + 3 = 3 \cdot 2^{29} - 1
        \end{gather*}
    \item
        $503 = 4(125) + 3$. Therefore this code belongs to a jump instruction.
        \begin{gather*}
            125 = 2^1(2 \times 31 + 1) - 1 \implies 125 = \pi(1, 31) \\
            1 = 2^1(2 \times 0 + 1) - 1 \implies 1 = \pi(1, 0) \\
            \implies 125 = \zeta(2, 1, 32)\\
            \implies \inv \beta(503) = J(2, 1, 32)
        \end{gather*}
    \item 
        First we compute the code for each instruction, then we use them to compute the program code:
        \begin{gather*}
            \beta(T(3, 4)) = 4(\pi(3 -1, 4 -1)) + 2 = 4(\pi(2, 3)) + 2 = 4 \cdot 27 + 2 = 110 \\
            \beta(S(3)) = 4(3-1) + 1 = 9 \\
            \beta(Z(1)) = 4(1 - 1) = 0 \\
            \tau(111, 9, 0) = 2^{110} + 2^{120} + 2^{121} - 1
        \end{gather*}
    \item 
        $100 = 2^0 + 2^2 + 2^5 + 2^6 - 1$. Thus $\inv \tau(100) = (0, 1, 2, 0)$
        \begin{gather*}
            \inv \beta(0) = Z(1)\\
            \inv \beta(1) = S(1) \\
            \inv \beta(2) = T(1, 1) \\
            \inv \beta(0) = Z(1)
        \end{gather*}
        Thus the program is: $Z(1), S(1),  T(1, 1),  Z(1)$.
\end{enumerate}