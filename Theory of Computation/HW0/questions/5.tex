کافیست رفتار ماشین تورینگ روی نوار را با این ماشین شبیه‌سازی کنیم.
فرض کنید نوار ماشین تورینگ به فرم زیر باشد به طوری که head روی خانه $x_i$ باشد:
\begin{center}
    \begin{tabular}{ |c|c|c|c|c|c|c|c|c| } 
     \hline
      $\dots$ & $\sqcup$ & $x_n$ & $\dots$ & $x_i$ & $\dots$ & $x_1$ & $\sqcup$ & $\dots$ \\
     \hline
    \end{tabular}
\end{center}

آنگاه صف به شکل زیر باشد:

\newpage

\begin{center}
    \text{Head Queue}\\
    \begin{tabular}{|c|} 
     \hline
     $x_i$ \\ 
     \hline
     $x_{i - 1}$ \\
     \hline
     $\vdots$ \\
     \hline
     $\$$ \\
     \hline
     $\vdots$ \\
     \hline
     $x_{i + 1}$ \\
     \hline
    \end{tabular}
\end{center}

حال دو حالت داریم:
\begin{enumerate}
    \item اگر دستور $x_i \rightarrow A, L$ باشد داریم:
        \begin{enumerate}[label=\arabic*.]
            \item pop \\
            \item A push
        \end{enumerate}
            همین عمل را روی صف عملی میکند و داریم:
            \begin{center}
                \begin{tabular}{|c|} 
                 \hline
                 $x_{i - 1}$ \\
                 \hline
                 $\vdots$ \\
                 \hline
                 $\$$ \\
                 \hline
                 $\vdots$ \\
                 \hline
                 $x_{i + 1}$ \\
                 \hline
                 $A$ \\
                 \hline
                \end{tabular}
            \end{center}
            تنها حالتی مشکل‌ساز است که بعد از این عملیات، علامت $\$$ روی ابتدای صف قرار بگیرد.
            این یعنی روی نوار تورینگ از ورودی رد شده‌ایم و به $\sqcup$ رسیده‌ایم.
            در اینصورت ابتدا $\$$ را پاپ کرده و سپس به ترتیب $\$, \sqcup, \#$ را پوش میکنیم. سپس یکبار تمام اعضای روی صف را پاپ کرده و بلافاصله پوش میکنیم تا به $\#$ برسیم.
            و با پاپ کردن آن به مرحله مورد نظر در تورینگ رسیده‌ایم.
    \item اگر دستور $x_i \rightarrow A, R$ باشد داریم:
            \begin{enumerate}[label=\arabic*.]
                \item $x_i$ pop
                \item $\#$ push
                \item A push
            \end{enumerate}
            سپس وارد یک زیرماشین با stateهای به نام الفبای نوار تورینگ میشویم. به طوریکه اگر در state یک حرف باشیم
            با خواندن حرف بعدی، ابتدا حرف state فعلی را پوش میکنیم. سپس حرف خوانده شده را پاپ میکنیم و به state حرف خوانده شده میرویم.
            هنگامی که به $\#$ رسیدیم ابتدا $\#$ را پاپ میکنیم و بلافاصله $\#$ را پوش میکنیم. سپس حرف state فعلی را پوش میکنیم و از زیرماشین خارج میشویم.
            سپس مثل بخش قبلی یکدور تمام حروف صف را پاپ و بلافاصله پوش میکنیم تا به $\#$ برسیم.
            $\#$را که پاپ کنیم به موقعیت ماشین تورینگ بعد از دستور ذکر شده میرسیم.
            اما مانند بخش قبلی اگر بعد از انجام مراحل در head استک مقدار $\$$ باشد یعنی از ورودی رد شدیم. در این حالت $\$$ را پاپ میکنیم و بلافاصله پوش میکنیم و بعد یک $\sqcup$ پوش میکنیم.
            سپس یکدور تمام اعضای صف را پاپ و بلافاصله پوش میکنیم تا دوباره به $\$$ برسیم.
            آنگاه اگر $\$$ را نیز پاپ و پوش کنیم به $\sqcup$ در ابتدای استک میرسیم.
\end{enumerate}
پس هردو نوع عمل روی نوار ماشین تورینگ شبیه سازی شد و ثابت میشود که هر ماشین تورینگ را میتوان با این ماشین نیز شبیه سازی کرد.