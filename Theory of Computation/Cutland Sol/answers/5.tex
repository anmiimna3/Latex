\begin{enumerate}[label=]
    \item 
        First we construct power in a recursive manner, with $g(x, y, z) = z \times x$:
        \begin{gather*}
            p(x, 0) = 1 \\
            p(x, y + 1) = g(x, y, p(x, y)) = p(x, y) \times x \\
            \implies p(x, y) = x^y
        \end{gather*}
        Now we can construct $t$ with substitution:
        \begin{gather*}
            t(x, y) = 2^x 3^y
        \end{gather*}
        Now we construct $\pi_1$ and $\pi_2$.
        \begin{gather*}
            \pi_1(x) = \mu_{z < x}(\chi_{2^z \mid x}) \\
            \pi_2(x) = \mu_{z < x}(\chi_{3^z \mid x})
        \end{gather*}
        It is easy to see that $(x, y) = (\pi_1(t(x, y)), \pi_2(t(x, y)))$. \newline \newline
        Now we define $h(x)$ such that $h(x) = 2^{f(x)}3^{f(x + 1)}$, with $g(x, y) = t(\pi_2(y), \pi_1(y) + \pi_2(y))$:
        \begin{gather*}
            h(0) = 2^{f(0)} 3^{f(1)} = 3 \\
            h(y + 1) = g(y, h(y)) = t(\pi_2(h(y)), \pi_1(h(y)) + \pi_2(h(y)))
        \end{gather*}
        Now it is easy to see that $\pi_1(h(n)) = \pi_1(2^{f(n)}3^{f(n + 1)}) = f(n)$. This shows that $f(n)$ is primitive recursive.
\end{enumerate}