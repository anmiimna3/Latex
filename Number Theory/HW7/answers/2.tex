\begin{enumerate}[label=\ilabel]
    \item 
        First note that since $\alpha > 1$ then if we have $\alpha = [a_0; a_1, a_2, \dots]$, then $a_0 > 0$. Now consider $\frac{1}{\alpha}$:
        \begin{gather*}
            \frac{1}{\alpha} = \frac{1}{a_0 + \frac{1}{a_1 + \dots}} \\
            \implies \frac{1}{\alpha} = [0; a_0, a_1, \dots]
        \end{gather*}
        And since $a_0 > 0$ this is valid. Now if we have $[0; a_1, a_2, \dots, a_k]$ which is the $(k + 1)$-th fraction for $\frac{1}{\alpha}$, we can write:
        \begin{gather*}
            0 + \frac{1}{a_0 + \frac{1}{\frac{\vdots}{a_k}}} = \frac{1}{\frac{p_k}{q_k}}
        \end{gather*}
        Where $\frac{p_k}{q_k}$ is the $k$-th fraction for $\alpha$. This completes the proof.


    \item 
        Similar to the last part, we have $\alpha = [a_0; a_1, \dots, a_k]$ where $a_0 > 0$ since $\alpha > 1$. Note that the fraction is finite since $\alpha$ is a rational number. Now we have:
        \begin{gather*}
            \frac{1}{\alpha} = \frac{1}{a_0 + \frac{1}{a_1 + \dots}}
        \end{gather*}
        Which shows that $\frac{1}{\alpha} = [0; a_0, a_1, \dots, a_k]$.
\end{enumerate}