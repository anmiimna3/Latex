\begin{enumerate}[label=]
    \item 
        We use the fact that $2 < \sqrt{7} < 3$:
        \begin{gather*}
            \begin{split}
                \frac{\sqrt{7} + 1}{2} & = 1 + \frac{\sqrt{7} - 1}{2} = 1 + \frac{1}{\frac{2}{\sqrt{7} - 1}} \\
                & = 1 + \frac{1}{\frac{2(\sqrt{7} + 1)}{7 - 1}} = 1 + \frac{1}{\frac{\sqrt{7} + 1}{3}} \\
                & = 1 + \frac{1}{1 + \frac{\sqrt{7} - 2}{3}} = 1 + \frac{1}{1 + \frac{1}{\frac{3}{\sqrt{7} - 2}}}
            \end{split}
        \end{gather*}
        And we have:
        \begin{gather*}
            \frac{3}{\sqrt{7} - 2} = \frac{3(\sqrt{7} + 2)}{7 - 4} = \sqrt{7} + 2 = 4 + \sqrt{7} - 2 \\
            = 4 + \frac{7 - 4}{\sqrt{7} + 2} = 4 + \frac{1}{\frac{\sqrt{7} + 2}{3}} = 4 + \frac{1}{1 + \frac{\sqrt{7} - 1}{3}}
        \end{gather*}
        Which we can write:
        \begin{gather*}
            \frac{\sqrt{7}- 1}{3} = \frac{1 }{\frac{3}{\sqrt{7} - 1}} = \frac{1}{\frac{3(\sqrt{7} + 1)}{7 - 1}} = \frac{1}{\frac{\sqrt{7} + 1}{2}}
        \end{gather*}
        Which is what we started with, thus we can write:
        \begin{gather*}
            \frac{\sqrt{7} + 1}{2} = [\overline{1, 1, 4, 1}]
        \end{gather*}
\end{enumerate}