\begin{enumerate}[label=]
    \item 
    First we define a function that checks if a number is prime.
    \begin{gather*}
        p(x) = \begin{cases}
            1 \ \ \ \ x \text{ is prime} \\
            0 \ \ \ \ x \text{ is not prime}
        \end{cases}
    \end{gather*}
    Suppose $E(x, y) = \chi_{x = y}$ and let $d(m, n) = \chi_{m \mid n}$ from problem 6.
    \begin{gather*}
        p(x) = E(x, \mu_{z \le x}((\chi_{z > 1}) \times d(z, x)))
    \end{gather*}
    This function returns 1 if smallest devisor of $x$, is $x$ itself. Otherwise it returns 0. \newline
    Now we define a recursive function that for any given $x$ returns the number of primes between $0$ and $x$.
    \begin{gather*}
        h(0) = 0 \\
        h(y + 1) = g(y, h(y)) = h(y) + p(y + 1) 
    \end{gather*}
    And at last we define function $prime(n)$:
    \begin{gather*}
        prime(n) = \mu_{z}(h(z) = n)
    \end{gather*}
    And for bounded minimalization we can use factorial functions, using $g(x, y) = y \times (x + 1)$:
    \begin{gather*}
        f(0) = 1 \\
        f(y + 1) = g(y, f(y)) = f(y) \times (y + 1) \\
        f(x) = x!
    \end{gather*}
    now we can create $prime(n)$ in a recursive manner:
    \begin{gather*}
        prime(0) = 0\\
        prime(y + 1) = g(y, prime(y)) = \mu_{z < prime(y)! + 2}(h(z) = y + 1)
    \end{gather*}
    This guarantees that $z$ searches for at least $prime(y)! + 1$ which is not devisible by first $y$ primes.
    thus the $y + 1$'th prime is lesser than or equal to $prime(y)! + 1$.
\end{enumerate}