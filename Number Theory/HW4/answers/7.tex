\begin{enumerate}[label=]
    \item 
        Similar to the Problem 5, suppose $Ord_{2p + 1}(a) = r$. $r$ can have values 1, 2, $p$ and $2p$. Since $2 \ne 1$ then $r \ne 1$. If $r = 2$ then $2^2 \ezmod{2p + 1} 1$ This shows that $2p + 1 = 3$. Which is a contradiction. If $r = p$ then we would have:
        \begin{gather*}
            2^p \ezmod{2p + 1} 1 \implies (-2)^p \ezmod{2p + 1} -1 \implies Ord(-2) \ne p\\
            -2 \ne 1 \implies Ord(-2) \ne 1 \\
            (-2)^2 = 4 \ezmod{2p + 1} 1 \implies p = 1 \implies Ord(-2) \ne 2
        \end{gather*}
        This shows that $Ord(-2) = 2p$ and thus $-2$ is a primitive root of $2p + 1$. Which we know is not true. Therefore $r = 2p$ and 2 is a primitive root of $2p + 1$.
\end{enumerate}