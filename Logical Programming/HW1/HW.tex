% !TeX program = xelatex

\documentclass[10pt,a4paper]{article}
\usepackage[inline]{enumitem}
\usepackage{pgfplots}
\usetikzlibrary{calc}
\newcommand{\drawaline}[4]{
	\draw [extended line=1cm,stealth-stealth] (#1,#2)--(#3,#4);
}
% \usepackage{unicode-math}
\usepackage{float}
\usepackage{tikz}
\usepackage{ifthen}
\usetikzlibrary {positioning}
%\usepackage {xcolor}
\definecolor {processblue}{cmyk}{0.96,0,0,0}

\usetikzlibrary{automata,arrows.meta}

\usepackage{commons/course}
\usepackage{hyperref}
\usepackage{multirow}
% \usepackage{newtxmath}
\usepackage{relsize}
\usepackage{graphicx}
\usepackage{neuralnetwork}
\usepackage{ifthen}
% \usepackage{relsize}

\hypersetup{
	colorlinks=true,
	linkcolor=cyan,
	filecolor=blue,      
	urlcolor=magenta,
	}
	
% \defaultfontfeatures{Scale=MatchLowercase}

%\hidesolutions
\newboolean{link}
\setboolean{link}{false}


\شروع{نوشتار}

برنامه‌سازی منطقی - بهار ۰۳-۰۴ \hfill سری ۱ \vspace{0.3cm} \\ 
امین ذوالفقاریان - ۶۱۰۳۰۰۱۱۸

\hrulefill

\newtheorem{problem}{سوال}

%%%
\newboolean{hideAnswers}
\setboolean{hideAnswers}{false}


% \newcommand{\answer}[2]{\href{#1}{\textbf{(#2)}}}
\newenvironment{answer}{\textbf{اثبات.} \hspace{.4cm}}{}

\newcommand{\customquestion}[1]{%
\begin{problem}
	\fontsize{13}{13.5}{
		\input{questions/#1.tex}
	}
\end{problem}
\begin{answer}
	\input{answers/#1.tex}
\end{answer}
}

% For question i, create files questions/i.tex and answers/i.tex and then use the command "\customquestion{question name}{score}{i}" to bring the question here. 

\customquestion{1}
\hrule
\customquestion{2}
\hrule
\customquestion{3}



\پایان{نوشتار}
