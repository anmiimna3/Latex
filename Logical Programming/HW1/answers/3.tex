\begin{enumerate}[label= \textbf{(\alph*)}]
    \item 
        ترم 
        $F = \lambda x. x S$ 
        را در نظر بگیرید. از آنجایی که  $S$ بسته است پس $F$ هم بسته است. از طرفی طبق قضیه نقطه‌ ثابت وجود دارد $t$ به گونه‌ای که داشته باشیم:
        $Ft = t$. به بیان بهتر:
        \begin{gather*}
            t = Ft = (\lambda x. x S) t = t S
        \end{gather*}
        تنها باقی میماند که نشان دهیم 
        $t$
        بسته است. برای این نیز فرض کنید این نقطه‌ی ثابت با کمک 
        کامبینیتور $Y$ بدست آمده است. 
        در این صورت چون $Y$ و $S$ و $F$ بسته هستند، پس $t$ نیز بسته است.
    \item
        دوباره تابع $F = \lambda y. y (\lambda x. x ) s$ را در نظر بگیرید. و فرض کنید $t'$ نقطه‌ی ثابتی برای آن باشد که به کمک کامبینیتور $Y$ بدست آمده است. دقت کنید که مشابه استدلال قبلی $t'$ بسته است.
        حال داریم:
        \begin{gather*}
            t' = F t' = (\lambda y . y (\lambda x. x) s) t' = t' (\lambda x . x) s \\
            \implies t's = t' (\lambda x. x) ss
        \end{gather*}
\end{enumerate}