\begin{enumerate}[label=]
    \item 
		We use the fact that in every tree $G$, for any two vertices $u, v \in V(G)$, there exists a unique path from $u$ to $v$. Now since $e \in E(\overline{G})$, then $u$ and $v$ are not connected in $G$. Let $P$ be the path between $u$ and $v$ in $G$. We know that $e$ is not in $P$, hence $P + e$ is a cycle in $G + e$. 

        Now suppose that $C$ is a cycle in $G + e$. Since $G$ does not contain any cycle, then $e$ is a part of $C$. Therefore $u$ and $v$ are two consecutive vertices in $C$. Since $C - e$ is contained in $G$, then it is the unique path from $u$ to $v$, and $C$ is the same path we introduced earlier. This shows that exists only 1 cylce in $G + e$.
\end{enumerate}