کارل $n$ کارت با تمام شماره‌های $1$ تا $n$ را دارد و به ترتیب رندومی روی میزش در یک ردیف قرار گرفته‌اند.
به یک جفت $(a, b)$ یک نابه‌جایی میگوییم، اگر $a < b$ باشد ولی $a$ سمت راست $b$ باشد.
ابتدا کارت شماره‌ی 1 را انتخاب کرده و آنرا به مکان قرینه‌ش نسبت به وسط منتقل میکنیم.
یعنی اگر این کارت سوم از سمت چپ باشد، آنرا به کارت سوم از سمت راست منتقل میکنیم و بقیه شیفت میخورند.
این کار را برای تمام اعداد 1 تا $n$ انجام میدهیم.
ثابت کنید بعد از این $n$ مرحله، تعداد نابه‌جایی‌ها تغییری نمیکند.
\textbf{(ناوردایی)}
\href{https://artofproblemsolving.com/community/c5h1630186p10232393}{\textbf{(2018)}}
\begin{gather*}
    3,1,4,2\to 3,4,1,2\to 2,3,4,1\to 2,4,3,1\to 2,3,4,1
\end{gather*}
