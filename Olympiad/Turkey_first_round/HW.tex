% !TeX program = xelatex

\documentclass[10pt,a4paper]{article}
\usepackage[inline]{enumitem}
\usepackage{pgfplots}
\usetikzlibrary{calc}
\newcommand{\drawaline}[4]{
	\draw [extended line=1cm,stealth-stealth] (#1,#2)--(#3,#4);
}
\usepackage{float}
\usepackage{tikz}
\usepackage{ifthen}
\usetikzlibrary {positioning}
%\usepackage {xcolor}
\definecolor {processblue}{cmyk}{0.96,0,0,0}

\usetikzlibrary{automata,arrows.meta}

\usepackage{commons/course}
\usepackage{hyperref}
\usepackage{multirow}
\usepackage{newtxmath}
\usepackage{relsize}
\usepackage{graphicx}
\usepackage{neuralnetwork}
\usepackage{paralist}
\usepackage{graphicx}
\graphicspath{{./images/}}

\hypersetup{
	colorlinks=true,
	linkcolor=cyan,
	filecolor=blue,      
	urlcolor=magenta,
}


%\hidesolutions

\شروع{نوشتار}

\begin{center}
	سوالات شمارش مرحله ۱ ترکیه
\end{center}

\hrulefill
%%%
\newboolean{hideAnswers}
\setboolean{hideAnswers}{false}
\newtheorem{problem}{سوال}
\newenvironment{solution}[1][\it{پاسخ}]{\textbf{#1. } }{$\square$}

% \newcommand{\customquestion}[1]{
% \textbf{سوال #1- }\input{questions/#1.tex}
% \newline
% \newline
% \phantom{سوال ۱۰-}\textit{پاسخ: }\input{answers/#1.tex}
% \newline
% \newline
% }

\newcommand{\customquestion}[1]{
	\begin{problem}
		\input{questions/#1.tex}
	\end{problem}
	\begin{solution}
		\input{answers/#1.tex}
	\end{solution}
}


% For question i, create files questions/i.tex and answers/i.tex and then use the command "\customquestion{question name}{score}{i}" to bring the question here. 


\customquestion{1}
\customquestion{2}
\customquestion{3}
\customquestion{4}
\customquestion{5}
\customquestion{6}
\customquestion{7}
\customquestion{8}
\customquestion{9}
\customquestion{10}
\customquestion{11}
\customquestion{12}
\customquestion{13}
\customquestion{14}









\پایان{نوشتار}
