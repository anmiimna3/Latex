\begin{enumerate}[label=\textbf{(\textit{\roman*})}]
    \item 
        We can rewrite the group action:
        \begin{gather*}
            a \in \ZZ, x \in \ZZ / 6\ZZ \\
            a.x = a + x \pmod 6
        \end{gather*}
        Then it is easy to see that this action is transitive. Since for any two $x, y \in \ZZ / 6\ZZ$ we can use $x - y \in \ZZ$ to see that they are in the same orbit.
        \begin{gather*}
            (x - y).y = x - y + y = x \overset{6}{\equiv} x \\
            x \in O_y 
        \end{gather*}
        And it is easy to see that if $x \in G_0$:
        \begin{gather*}
            0 + x \overset{6}{\equiv} 0 \implies x \overset{6}{\equiv} 0
        \end{gather*}
        Thus $G_0=\{6k| k \in \ZZ\}$.
    \item  
        Let $A \in \stab(I_n)$. Therefore $AI_n = I_n$ and also we know that for any $A$ we have $AI_n = A$. This shows that $A = I_n$ and $\stab(I_n) = I_n$. \newline
        As for orbit if $A$ and $B$ have the same orbit then there exists a $C \in GL_n(\RR)$ such that $CA = B$. Now to show that they have the same kernel space:
        \begin{gather*}
            Ax = 0 \implies CAx = 0 \implies Bx = 0 \implies ker(A) \subset ker(B) \\
            Bx = 0 \implies CAx = 0 \implies \inv C CAx = 0 \implies Ax = 0 \implies ker(B) \subset ker(A) \\
            \implies ker(B) = ker(A)
        \end{gather*}
        Therefore if two matrices have the same orbit then they have the same kernel space.
        We also know that if two matrices have the same kernel space Therefore There exists an invertible matrix $Q$ such that $QA = B$. This shows that if two matrices have the same kernel space then they are in the same orbit.
\end{enumerate}