\begin{enumerate}[label=\ilabel]
    \item 
    $(\Rightarrow)$ Suppose that $M$ is irreducible. We know by definition $M\neq 0$. Taking some nonzero $m\in M$, we see that $Rm$ is a nonzero submodule of $M$, and so $Rm = M$. This proves that $M$ is generated by any nonzero element.
    
    ($\Leftarrow$) Suppose $M$ is a cyclic $R$-module with generator $a \ne 0$. Then if $I \subseteq R$ is an ideal of $R$, then $Ia \subseteq Ra = M$ is a proper submodule of $M$. I think for this statement to hold, we need to have that $M$ is cyclic if for any element $m \in M$, we have $M = Rm$. Then see that for some submodule $N \subseteq M$, take $b \in N$, then $M = Rb \subseteq N$. Which shows that $N = M$, and thus $M$ is irreducible. 

    For all irreducible $\ZZ$-modules we have all $\ZZ / p\ZZ$ as $\ZZ$-modules. Which is easy to see that they are irreducible since they are all cylic and don't have any subgroup.

    \item
        Take $R = Z[x, y]$. It is easy to see that $R$ is a cyclic $R$-submodule, since its generated with $1$. And we know that $\braket{x, y}$ is an ideal of $R$ which can't be generated with only 1 element. Thus is not cyclic.
        
    \item
        Let $N$ be a submodule of $M$, a cyclic $R$-module generated with $a \ne 0$. Then $N = \big(r_i a\big)_{i \in J}$ for some index set $J$. Now suppose $i, j \in J$ and $r \in R$, since $N$ is a submodule, then we have that $r_i a + r r_j a = (r_i + r r_j) a \in N$.
        Therefore $r_i + r r_j = r_k$ for some $ k \in J$. Which is describing and ideal of $R$. Now if we have some $r_i a = 0$ and $r_j a = 0$, then it is easy to see that for any $r \in R$, $rr_i a + r_j a = 0$. which implies that $(r r_i + r_j) a = 0$. Therefore all coefficients of $a$ such that their product is 0, form another ideal. This proves that we can write $N = \frac{I}{K} a$ for some $K \subseteq I \subseteq R$ ideals of $R$. 
\end{enumerate}