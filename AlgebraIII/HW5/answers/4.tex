\begin{enumerate}[label=]
    \item 
        \textbf{(a)}
        Let $r \in Ann_\ZZ(M)$. Then for any $(a, b, c) \in M$ we have $r(a, b, c) = (0, 0 , 0)$. Let $a = 1_{\ZZ_{24}}, b = 1_{\ZZ_{15}}, c = 1_{\ZZ_{50}}$. Therfore we have:
        \begin{gather*}
            r(a, b, c) = (r, r, r) = (0, 0, 0)
        \end{gather*}
        This shows that:
        \begin{gather*}
            \begin{rcases}
                24 \mid r \\
                15 \mid r \\
                50 \mid r
            \end{rcases} \implies 2^3 \times 3 \times 5^2 = 600 \mid r \implies r \in 600\ZZ
        \end{gather*}
        And since for any $r \in 600\ZZ$ we have:
        \begin{gather*}
            600(a, b, c) = (600a, 600b, 600c) = (0, 0, 0)
        \end{gather*}
        Which implies $r \in Ann_\ZZ(M)$, then we have $Ann_\ZZ(M) = 600\ZZ$.
    \item
        \textbf{(b)}
        If $(a, b, c) \in Ann_M(2\ZZ)$ then we have:
        \begin{gather*}
            \begin{split}
                2 \in 2\ZZ : 2(a, b, c) & = (2a, 2b, 2c) = (0, 0, 0) \\
                \implies a & \in \{0, 12\} \cong \ZZ_2 \\
                b & \in \{0\} \cong \ZZ_1 \\
                c & \in \{0, 25\} \cong \ZZ_2
            \end{split}
        \end{gather*}
        This shows that $Ann_M(2\ZZ) \cong \ZZ_2 \times \ZZ_2$. It is easy to see that for any $2i \in 2\ZZ$ it works.
\end{enumerate}