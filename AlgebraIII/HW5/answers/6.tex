\begin{enumerate}[label=]
    \item 
        First we show that $eM$ and $(1 - e)M$ is submodule of $M$:
        \begin{gather*}
            x, y \in M \implies ex, ey \in eM, r \in R \\
            r(ex) + ey = rex + ey = erx + ey = e(rx + y) \in eM
        \end{gather*}
        Note that we used $re = er$ and also that since $M$ is module then $rx + y \in M$. Similarly for $(1 - e)M$ we have:
        \begin{gather*}
            \begin{split}
                r((1-e)x) + (1-e)y & = rx - rex + y - ey \\
                & = rx + y - (erx - ey) \\
                &= rx + y - e(rx - y) \in (1 - e)M
            \end{split}
        \end{gather*}
        It is obvious that since both $eM$ and $(1 - e)M$ are $R$ module, so is $eM \oplus (1 - e)M$. It remains to show an isomorphism:
        \begin{gather*}
            \varphi: M \to eM \oplus (1 - e)M \\
            \phantom{llll}m \mapsto (em, (1-e)m) \\
            \begin{split}
                \varphi(m + m') & = (e(m + m'),(1 - e) (m + m')) \\
                & = (em , (1 - e)m) + (em', (1 - e)m') \\
                & = \varphi(m) + \varphi(m'). \\
                \varphi(rm) & = (erm, (1 - e)rm) \\ 
                & = (rem, (r - re)m) \\
                & = (rem, r(1 - e)m) \\
                & = r(em, (1 - e)m) = r \varphi(m).
            \end{split}
        \end{gather*}
        This shows that $\varphi$ is a homomorphism. Now to show that it is an isomorphism we have to show that it is both surjective and injective.
        Suppose $em + (1 - e)m' \in eM \oplus (1 - e)M$.
        For any $m, m' \in M$ we know $em + (1 - e)m' \in M$. Thus we have:
        \begin{gather*}
            \begin{split}
                \varphi(em + (1 - e)m') & = \big(e(em + (1 - e)m'), (1 - e) (em + (1 - e)m') \big) \\
                & = \big((e^2m + (e - e^2) m'), (e - e^2)m + (1 - 2e + e^2)m'\big)
            \end{split}
        \end{gather*}
        Sicne $e^2 = e$:
        \begin{gather*}
            = \big(em, (1 - e)m'\big)
        \end{gather*}
        This gives us that $\varphi$ is surjective. For injective suppose $m, m' \in M$ where $m \ne m'$:
        \begin{gather*}
            \varphi(m) = \varphi(m') \\
            \implies (em, (1 - e)m) = (em', (1 - e)m') \\
            \implies em = em' , (1 - e)m = (1 - e)m' \\
            \implies e(m - m') = 0 , (1 - e)(m - m') = 0 \\
            \implies e(m - m') + (1 - e)(m - m') = 0 \\
            \implies m - m' = 0 \implies m = m'
        \end{gather*}
        Which is a contradiction. This gives us that $\varphi$ is bijective as well. Therefore $\varphi$ is an isomorphism and thus we have:
        \begin{gather*}
            M \cong eM \oplus (1 - e)M
        \end{gather*}
\end{enumerate}