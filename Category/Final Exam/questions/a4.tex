\begin{enumerate}
    \item 
        هر عضو از $\hat{\mathbf{2}}$ یک فانکتور از $\mathbf{2}$ به $\mathbf{Set}$ است. از آنجایی که دو عضو ابتدایی در $\mathbf{2}$ یکریخت هستند، پس به اشیای یکریختی نیز در $\mathbf{Set}$ میروند. و مورف‌های بین آنها نیز به یک همریختی و وارون آن میروند. از طرفی به هر مورف همریختی در $\mathbf{Set}$ میتوان یک فانکتور نسبت داد که $\mathbf{2}$ را به آن همریختی میبرد.
         پس هر فانکتور تنها با یک یکریختی در $\mathbf{Set}$ مشخص میشود. پس درواقع $\hat{\mathbf{2}}$ کتگوری تمام یکریختی‌های (توابع ۱-۱ و پوشا) داخل $\mathbf{Set}$ است.
         مورف‌های $\hat{\mathbf{2}}$ هم تمام تبدیلات طبیعی بین این فانکتورها هستند.
        هر تبدیل هم معادل یک جفت تابع بین اشیای یکریخت است که دیاگرام جابه‌جایی شود:
         \begin{center}
            \tikz{
                \node at (2, 2) (x) {$X$};
                \node at (2, 0) (x1) {$X'$};
                \node at (0, 2) (y) {$Y$};
                \node at (0, 0) (y1) {$Y'$};
                \path[<->]
                    (x) edge [anchor = west] node {$x$} (x1)
                    (y) edge [anchor=east] node {$y$} (y1);
                \path[->]
                    (y1) edge [anchor=north] node {$g$} (x1)
                    (y) edge [anchor = south] node {$f$} (x);
            }
         \end{center}
         که در آن $x$ و $y$ توابعی یک به یک پوشا هستند (وارون دارند) و با تعیین کردن $f$، تابع $g$ به صورت یکتا مشخص میشود به طوری که دیاگرام بچرخد.
    \item 
    \item 
\end{enumerate}