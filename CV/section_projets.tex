\sectionTitle{Projects and Researches}{\faLaptop}

\begin{projects}






    \project
    {Formalization of Category Theory in Coq}{2025}
    {\website{https://github.com/jwiegley/category-theory}{Source Github Repository}}
    {An attempt to study the formalization of Category Theory in Coq as a case study to deepen my understanding of proper implementation techniques in Coq.}
    {Category Theory, Coq}
    
    \project
    {Software Foundations Series}{2025}
    {\github{AmirFaridi-2002/Software-Foundations}}
    {Engaged with the 'Software Foundations' series to enhance proficiency in formal reasoning and proof development using Coq. Completed the first volume and all of its exercises, i.e. 'Logical Foundations'. Currently progressing through the third volume, 'Verified Functional Algorithms', focusing on the verification of functional algorithms. (The repository is private due to the book's policy against publicly sharing solutions to exercises.)  }
    {Coq, Formal Verification, Functional Programming, Algorithms}
    
    






 %    \project
 %    {Computability in Constructive Type Theory}{2024}
	% {\website{https://ps.uni-saarland.de/~forster/thesis/phd-thesis-yforster-screen.pdf}{Source Thesis Paper}}
	% {An attempt to study and understand the formalization of the concepts of computability in Coq proof assistant, started with the Type Theory and Formal Proofs by Rob Nederpelt \& Herman Geuvers, and continuing with the Interactive Theorem Proving and Program Development by Yves Bertot \& Pierre Casteran as prerequisites.  \website{https://github.com/AmirFaridi-2002/Coq-Journey}{My GitHub repository on the exercises of the Coq' Art}
 %    }
	% {Intuitionistic Logic, Lambda Calculus, Type Theory, Calculus of Inductive Constructions, Coq, Computability, Models of Computation}
    
	\project
	{Computational Neuroscience}{2024}
	{\github{https://github.com/AmirFaridi-2002/CNS}}
	{Topics covered neural biophysics, synaptic plasticity, network connectivity, dynamics, learning, and using these ideas on visual cortex. The course combined theoretical foundations with practical modeling to gain insights into brain computation.}
	{LIF, ELIF, AELIF, Synapses, Decision Making, STDP, RSTP, Lateral Inhibition, k-Winner-Takes-All, Homeostasis, Gabor, DoG, Spike Encodings, Feature Extraction, Image Processing}
				
	\project
	{Compiler Design}{2024}
	{\github{AmirFaridi-2002/Compiler}}
	{This repository contains the implementation of a custom compiler, developed as part of a project to explore language design and compilation techniques.}
	{ANTLR, Name Analysis, Type Analysis, Context Free Grammars}

    \project
	{MSDOS - Image Processing Using Assembly Language}{2024}
	{\github{AmirFaridi-2002/Assembly}}
	{The first folder in the repository contains my exploration of the MS-DOS v1.25 source code, specifically focusing on the COMMAND.ASM file. The code is written in assembly language.
                The other folder contains the homework for the course and the main project, a set of image-processing tools implemented as part of the Machine Language and Assembly course. The project applies low-level programming concepts to perform operations on images using assembly and system calls.}
	{MSDOS, NASM, System Calls, Image Processing}


    
    \project
	{Pyxcel}{2021}
	{\github{AmirFaridi-2002/Pyxcel}}
	{Pyxcel is an interpreter for a custom language designed to create and manipulate tables. The project also includes web scraping functionality and server-side data management.}
	{Regex, Threading, Socket, Pandas, BS4, Request, Pillow}






 %    \project
	% {Price Tracker }{2024}
	% {\github{AmirFaridi-2002/CC-Monitor}}
	% {This Python project monitors the prices of multiple cryptocurrencies and sends alerts through a Telegram bot when a specific price pattern is met. The project also generates charts for analysis and provides them to authorized admins.}
	% {ccxt, pandas, socket, threading, telegram bot API}

 %    \project
 %    {Basic Computer}{2023}
	% {\github{AmirFaridi-2002/BasicComputer}}
	% {This project is a complete implementation of the 16-bit Mano Machine, created using LogiSim. It utilizes basic memory and register modules, including components like the Accumulator (AC), Address Register (AR), and Program Counter (PC), to replicate the machine's functionality.}
	% {Accumulator, Computer Architecture, Digital Design}

    % \project
    % {Artificial Intelligence}{2024}
    % {\github{AmirFaridi-2002/Artificial-Intelligence}}
    % {This repository serves as a collection of my projects undertaken in the AI course. It will be periodically updated to include new projects as they are completed.}
    % {Search Algorithms}
 
 %    \project
 %    {Universal URM}{2024}
	% {\github{Private}} % https://github.com/AmirFaridi-2002/UTM
	% {This project involves implementing a Universal URM as part of my responsibilities as a teaching assistant for the theory of computation course. It's currently a work in progress, designated as a student assignment.}
	% {Universal Unlimited Register Machine}
\end{projects}